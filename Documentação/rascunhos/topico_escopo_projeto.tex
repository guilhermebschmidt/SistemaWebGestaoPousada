Desenvolvimento do Projeto


        Nesta seção, detalham-se as tecnologias, ferramentas e práticas adotadas para o desenvolvimento da aplicação web de gestão dos processos da pousada Chalés Água de Coco.


Escopo do Projeto 


O projeto visa automatizar os processos de gestão de reservas, quartos, hóspedes, finanças e produção de relatórios da pousada, com o objetivo de proporcionar à proprietária uma ferramenta eficiente e organizada para a realização de todas as operações essenciais do seu negócio. Além disso, busca mitigar os riscos associados ao modelo atual de administração, caracterizado por uma alta vulnerabilidade a erros humanos, dificuldade de atualização em tempo real e ausência de acessibilidade remota, decorrentes do seu caráter predominantemente manual.
Prevê-se que o produto final seja constituído por funcionalidades que podem ser organizadas em cinco módulos principais: módulo de gestão de reservas, módulo de gestão de hóspedes, módulo de gestão de quartos, módulo de gestão financeira e painel administrativo. Com uma interface intuitiva e integrada e desenvolvido exclusivamente para o uso de uma única usuária: a proprietária da pousada.
Entretanto, o escopo deste projeto não abrange funcionalidades como a integração com sistemas de pagamento ou serviços de hospedagem. Porém, vale-se destacar que o escopo prevê, em etapas futuras, a evolução do sistema para incluir integrações com serviços de mensageria e correio eletrônico, ampliando a automação dos serviços da pousada.
Por fim, Para a definição do escopo, a equipe responsável aplicou um conjunto de perguntas (questionário) à proprietária com o objetivo de compreender as regras de negócio e realizar o levantamento dos requisitos, funcionais e não funcionais, do sistema. 


Quadro X - Questionário usado para a elaboração do escopo do projeto 
	Pergunta
	Resposta
	Como os clientes normalmente fazem uma reserva hoje (telefone, WhatsApp, pessoalmente)?
	

Geralmente, por WhatsApp.
	Há um prazo mínimo ou máximo para fazer uma reserva (ex: com 1 dia de antecedência)?
	 Preferencialmente, antecipadamente. Nas plataformas coloco 2 dias de antecedência.


	A reserva é confirmada apenas com pagamento ou pode ser feita sem pagamento antecipado?
	Confirmada pelo pagamento de pelo menos metade do valor da reserva.
	É permitido cancelar uma reserva? Até quantas horas antes do check-in? Há cobrança de taxa?
	Sim!  Temos políticas de cancelamentos e remarcações.


	Um hóspede pode fazer mais de uma reserva ativa ao mesmo tempo?
	

Sim.


	Quais dados do hóspede são obrigatórios para fazer uma reserva? (ex: nome completo, CPF, telefone, e-mail)
	Nome completo, endereço completo,  CPF, telefone, e-mail.
	É comum haver reservas feitas por um responsável em nome de outros hóspedes?
	Sim.


	Há um limite de pessoas por quarto? Como isso é controlado?
	No check in.


	 


Qual é o horário padrão de check-in e check-out? Há tolerância?
	 Check in a partir das 16h até às 22h e checkout a partir das 8h até às 14h.  Depende de se tem entrada de outro hóspede em seguida.
	



O check-in pode ser feito antes do horário? E o check-out após o horário?
	Depende, se tiver saída de hóspede anterior.  Depende se tiver entrada em seguida.


	

Quem realiza o check-in e check-out? Você ou os funcionários?
	Eu ou sozinhos, com orientações minhas. 


	Há necessidade de gerar comprovante ou recibo após check-in ou check-out?
	Não.


	Domínio: Quartos
	Pergunta
	Resposta
	



A pousada possui quantos quartos? Como eles são classificados (ex: simples, casal, com ar)?
	16. Quartos simples simples para casal (com e sem ar), chalés com cozinha para até 4 pessoas, com e sem ar e flats, com e sem ar. 
	Há períodos em que quartos são bloqueados para manutenção?
	Sim.
	Um mesmo quarto pode ser reservado para diferentes hóspedes em dias seguidos?
	Sim.
	A pousada oferece serviços extras? 
	Não. 
	

Quais formas de pagamento são aceitas (Pix, cartão, dinheiro)?
	As três formas, porém no cartão tem taxa da operadora. 
	

Os pagamentos são feitos no check-in, no check-out ou antecipadamente?
	Metade na reserva e o restante na chegada
	

É necessário gerar comprovante ou recibo no sistema?
	Eu envio uma confirmação de reserva com todas as informações.
	 Você gostaria que o sistema enviasse confirmação automática de reserva por WhatsApp ou e-mail?
	Sim.


	Há interesse em receber alertas automáticos de check-in, check-out ou cancelamento?
	Sim.




	

Somente você vai usar o sistema ou os funcionários também?
	Somente eu.
	 O sistema será usado no computador, celular ou ambos?
	Em ambos.
	Quais relatórios são mais importantes no dia a dia? (ex: ocupação diária, reservas da semana, totais de pagamento)
	Ocupação diária, reservas da semana, totais de pagamento, período…


	É importante ter um histórico de cada hóspede e das reservas anteriores?
	Sim.


	Regras de negócio
        As regras de negócio são diretrizes que delimitam as relações entre os objetos do negócio, estabelecendo condições e restrições que orientam as empresas nas suas operações e processos internos (IBM, 2024).
Assim, o levantamento das regras de negócio da Pousada Chalés Água de Coco, resultado da análise das respostas fornecidas pela proprietária ao questionário aplicado pela equipe, foi fundamental para orientar a modelagem e o desenvolvimento das funcionalidades do sistema. Além de assegurar que a aplicação atenda as necessidades da pousada de forma coerente, segura  e eficiente, além de apoiar, no contexto de sua gestão, os processos de tomada de decisão. 




Regra
	Descrição 
	RN01
	A confirmação de uma reserva deve ocorrer mediante o pagamento de 50% do valor total.
	RN02
	Cancelamentos e remarcações são permitidos, sujeitos a regras e taxas específicas.
	RN03
	As reservas devem ser registradas com, no mínimo, 2 dias de antecedência da data de entrada.
	RN04
	Um mesmo hóspede pode ter mais de uma reserva ativa.
	RN05
	Para efetuar uma reserva, os seguintes dados do hóspede são obrigatórios: nome completo, endereço completo, CPF, telefone e e-mail
	RN06
	Um responsável pode realizar reservas em nome de outros hóspedes.
	RN07
	Cada quarto deve ter um número máximo de hóspedes.
	RN08
	Os horários padrão são: check-in das 16h às 22h; check-out das 8h às 14h.
	RN09
	A emissão de recibos ou comprovantes após check-in/out não é obrigatória.
	RN10
	Um  quarto pode ficar indisponível para manutenção.
	RN11
	É permitido reservar um mesmo quarto para hóspedes diferentes em datas seguidas.
	RN12
	Atualmente, não há oferta de serviços adicionais vinculados à reserva.
	RN13
	São aceitas as formas de pagamento: Pix, dinheiro e cartão (com taxa da operadora).
	RN14
	Deve ser enviada a confirmação da reserva para o cliente com todos os dados necessários.
	RN15
	Apenas a proprietária deve acessar as informações dos hóspedes e reservas.
	







Requisitos do Sistema
        Segundo Sommerville (2011, p.57), os requisitos de um sistema são as descrições do que o sistema deve fazer, os serviços que oferece e as restrições a seu funcionamento. Assim, considerando a importância desses elementos para a modelagem da aplicação web de gestão para a Pousada Chalés Água de Coco , foi realizado o levantamento dos requisitos funcionais e não funcionais. Esse processo é resultante da análise das respostas obtidas por meio do questionário aplicado à proprietária (). 
Requisitos funcionais
Os requisitos funcionais são aqueles que descrevem as funcionalidades e serviços do sistema web de gestão da pousada. Tais requisitos estão listados no quadro “”.
Requisitos Funcionais
	Código
	Descrição
	Prioridade
	Relacionamento com Regras de Negócio
	Domínio: Acesso e Autenticação
	RF01
	O sistema deve possuir um sistema de login para a proprietária acessar a aplicação de forma segura.
	



Alta
	



Não aplicável — requisito técnico essencial
	Domínio: Reservas
	

	

	RF02
	O sistema deve permitir o cadastro de reservas, associando um quarto a um período (data de check-in e check-out).
	



Alta
	



RN03; RN01
	RF03
	Uma reserva deve estar associada a um quarto disponível para que ela seja cadastrada.


	



Alta
	



RN11
	RF04
	O sistema deve exigir os dados pessoais do hóspede para que a reserva seja cadastrada: nome completo, endereço completo, CPF, telefone e e-mail.


	







Alta
	





RN05 
	RF05
	A proprietária deve conseguir reservar quartos para um cliente em nome de outra pessoa responsável, registrando os dados do hóspede e, opcionalmente, do responsável.
	







Média
	







RN06
	  RF06
	O sistema deve exigir o pagamento de 50% do valor da estadia para confirmar o cadastro da reserva (a ser pago no momento da reserva ou em um prazo definido). 
	







Média
	







RN01
	RF07
	O sistema deve permitir o registro da comprovação do pagamento.
	

Média
	

RN01 
	RF08
	A proprietária deve conseguir cancelar ou remarcar uma reserva, com possível registro do motivo.
	



Média
	



RN02 
	RF09
	A proprietária deve poder visualizar todas as reservas, com detalhes do hóspede, quarto reservado e período.
	



Alta
	



RN15 
	RF10
	A proprietária deve conseguir cadastrar mais de uma reserva no nome de um mesmo cliente.
	



Média
	



RN04 
	RF11
	A proprietária deve conseguir reservar um mesmo quarto para diferentes clientes em datas seguidas, respeitando os horários de check-in e check-out configurados para o quarto.
	







Alta
	







RN11
	RF12
	A proprietária deve conseguir acessar o histórico de reservas de um cliente.
	

Média
	

RN15
	RF13
	O sistema deve mudar o status do quarto para ocupado após a realização do check-in.
	Alta
	

Não aplicável — requisito técnico essencial
	RF14
	O sistema deve enviar uma notificação para o hóspede após a confirmação da reserva.
	



Baixa
	



Não aplicável — requisito de suporte à operação


	Domínio: Quartos 
	

	

	RF15
	O sistema deve permitir o cadastro de novos quartos, incluindo informações como número/nome do quarto, capacidade (número de hóspedes), tipo (ex: chalé, simples solteiro, simples casal, etc.), e preço por noite.
	









Alta
	









Não aplicável — requisito técnico essencial
	RF16
	A proprietária deve conseguir editar as informações dos quartos já cadastrados.
	



Alta
	

Não aplicável — requisito técnico essencial
	RF17
	A proprietária deve poder visualizar todos os quartos cadastrados.
	

Alta
	

Não aplicável — requisito de suporte à operação
	RF18
	O sistema deve permitir a visualização dos quartos disponíveis no período de tempo selecionado para a reserva
	





Média
	





RN03; RN11
	RF19
	A proprietária deve conseguir mudar o status de um quarto (ex: disponível, indisponível, em manutenção).


	





Alta
	





RN10 
	Domínio: Financeiro
	

	

	RF20
	A proprietária deve poder registrar as despesas da pousada, categorizando-as (ex: manutenção, limpeza, contas de consumo), especificando a data, o valor, a categoria e uma descrição da despesa
	









Média
	







Não aplicável — requisito técnico essencial


	RF21
	A proprietária deve conseguir cadastrar gastos fixos e gastos variáveis.
	

Média
	Não aplicável — requisito técnico essencial


	RF22
	A proprietária deve poder registrar receitas, associando-as a uma reserva ou a outras fontes de receita, especificando a data, o valor e uma descrição da receita.
	







Média
	







RN13 
	RF23
	O sistema deve permitir que a proprietária visualize todas as transações financeiras (receitas e despesas) em um determinado período.
	





Média
	



Não aplicável — requisito de controle gerencial
	RF24
	O sistema deve permitir a filtragem das transações por tipo (receita/despesa), data e categoria.
	



Média
	

Não aplicável — requisito técnico essencial
	RF25
	O sistema deve ser capaz de gerar um balanço financeiro simples para um período selecionado, mostrando o total de receitas, o total de despesas e o saldo.
	





Média
	





Não aplicável — requisito técnico essencial
	

Requisitos não funcionais 
Os requisitos não funcionais descrevem, por sua vez, as restrições e características de qualidade que devem ser aplicadas às funções e serviços prestados pelo sistema web. Estes requisitos estão listados no quadro “”.
Usabilidade
	A interface do sistema deve ser intuitiva, responsiva (compatível e adaptada tanto para dispositivos desktop quanto mobile)  e de fácil utilização, de modo que as tarefas essenciais da gestão da pousada sejam realizadas de forma eficiente e com mínimo esforço de aprendizado pela proprietária.  Para isso, deve-se adotar os princípios de interface amigável como a priorização da simplicidade e da clareza, padrões de interface consistentes e acessíveis.
	O sistema deve fornecer mensagens de feedback claras, objetivas e contextualizadas para  todas as ações  realizadas pela usuária, como confirmação de reserva (exemplo: reserva efetuada com sucesso) ou notificações de erros (exemplo: falha ao cadastrar um quarto), garantindo uma interação segura e  satisfatória .
	Performance
	O sistema deve apresentar um tempo de resposta baixo, com carregamento das páginas e execução de ações da proprietária entre 2 e 3 segundos, para garantir uma navegação fluida .
	O sistema deve ser capaz de lidar com a carga de trabalho estimada – desde o registro e a consulta simultânea de dados à gestão de múltiplas reservas– sem degradação significativa no desempenho, este que deverá se manter estável mesmo em períodos de maior demanda, considerando o perfil sazonal do negócio. 
	Segurança
	O sistema deve garantir a segurança das informações da pousada e dos hóspedes através da implementação de mecanismos robustos de autenticação e autorização, de forma a assegurar que apenas a usuária autorizada consiga acessar ou alterar dados na aplicação.
	Os dados sensíveis devem ser protegidos conforme as melhores práticas propostas pela LGPD (Lei Geral de Proteção de Dados Pessoais), incluindo: utilização de criptografia para proteger dados em trânsito (HTTPS) e em repouso, implementação de políticas de autenticação robusta e minimização da coleta de dados.
	Confiabilidade
	O sistema deve estar disponível e funcionando corretamente por pelo menos  99% do tempo, a fim de garantir que a proprietária tenha acesso ao sistema sempre que necessário, inclusive nos períodos com maior movimento de hóspedes na pousada.
	O sistema deve implementar mecanismos de tratamento de erros para que falhas e perdas de dados sejam prevenidas.  
	O deploy da aplicação deve ser realizado em uma infraestrutura de nuvem (Amazon EC2),  a fim de proporcionar maior estabilidade, flexibilidade à aplicação e permitir que possíveis atualizações e manutenções tenham impacto mínimo para a usuária.
	Escalabilidade
	Embora o sistema, inicialmente , seja voltado para uma única usuária, a arquitetura deve ser projetada de forma a permitir futuras expansões no número de usuários e funcionalidades sem grandes refatorações.
	Documentação
	O sistema deve possuir uma documentação completa, objetiva e atualizada, incluindo código- fonte, a arquitetura da aplicação, os fluxos de uso e as especificações de APIs possivelmente integradas . 
	A documentação deve estar versionada e organizada em no repositório Git– o GitHub, este que deve ser utilizado no controle de versão da aplicação  e colaboração entre os membros da equipe. 
	O desenvolvimento deve seguir as boas práticas de codificação e padrões recomendados para aplicações Django, a fim de assegurar a manutenibilidade, extensibilidade e integridade do sistema ao longo do seu ciclo de vida.
	