\section {Revisão da Literatura}
Neste tópico é revisado o histórico do uso da tecnologia na gestão de negócios de hotelaria na literatura científica, com o objetivo de contextualizar e destacar a relevância do projeto.

\subsection{Histórico do Turismo e da Hospitalidade} 
A atividade de turismo é o ato de uma pessoa de se deslocar para um local diverso da sua residência por diferentes motivações, desde econômicas, que datam a antiguidade com as grandes viagens exploratórias dos povos antigos, até em entretenimento.  A partir do século XVII, os avanços tecnológicos em diferentes setores mudaram profundamente o modo de vida do homem e suas relações sociais, o que fortaleceu o turismo de entretenimento, que se tornou um forte pilar econômico para diferentes regiões do mundo (IGNARRA, 2013).

No Brasil em 2024, segundo a Federação do Comércio de Bens, Serviços e Turismo do Estado de São Paulo (FecomercioSP), o setor do turismo cresceu 4,3\% em relação a 2023, o que gerou um faturamento de 207 bilhões de reais. Sendo que, o estado de São Paulo representou 34\% do rendimento total. Fato que expõe a importância do turismo para a economia  do país, além da força das regiões turísticas do estado de São Paulo. 

Historicamente, como consequência do crescimento do turismo, houve o aumento de estruturas físicas e sociais para atender as necessidades dos viajantes, dentre elas, os meios de hospedagem. Dentro do serviço de hospedagem existem diferentes tipos de estrutura, entre elas a pousada. Segundo Zanella e Angeloni,  a pousada é um ambiente pequeno e de arquitetura simples que presta serviços de hospedagem, alimentação e lazer, de forma criativa e personalizada. 

\subsection{A Gestão Hoteleira e o Impacto da Tecnologia}
Uma pousada, como outros negócios, precisa promover um processo de gestão de seus serviços com o objetivo de atender as expectativas dos clientes, melhorar a satisfação e retê-los (ZANELLA; ANGELONI, 2006). A gestão de um empreendimento de hospedagem é geralmente departamentalizada em: Hospedagem, Marketing, Finanças e Contabilidade, Administração e Segurança, Alimentos e Bebidas; e Eventos e Serviços Diversos. Essa estrutura varia de acordo com a categoria da hotelaria, mas que é fundamental para a coordenação das atividades, atração de hóspedes e geração de lucro (MARTINS; GONDIM, 2011).

De acordo com Sidônio, com o aumento da concorrência e as mudanças nos hábitos dos clientes decorrentes da globalização, para atingir seus objetivos, uma empresa hoteleira deve sempre buscar informações para o embasamento das tomadas de decisões tanto de planejamento quanto de inovação dos seus serviços. Dessa forma,  a gestão hoteleira exige que o gestor esteja atento às constantes mudanças nas tendências do mercado contemporâneo, devendo ser ágil e prático nas suas tomadas de decisão (MAURÍCIO E RAMOS, 2011). Nesse contexto, as tecnologias da informação surgem nesse setor como ferramentas capazes de auxiliar no aumento da competitividade da empresa hoteleira (BUHALIS, 1998). 

A tecnologia afeta a competitividade das empresas através do fornecimento de benefícios estratégicos. Isso porque, essa ajuda o acesso a informações que auxiliam as empresas na customização dos seus produtos e serviços,  na garantia de preços competitivos, diminuição dos custos associado a um aumento de eficiência e na construção de relacionamentos mais próximos com fornecedores e clientes. Assim, o setor do turismo, sendo uma área que precisa estar sempre se adaptando às necessidades do mercado e do cliente, é altamente beneficiado pelo uso das novas tecnologias  (BUHALIS, 1998).

Dessa forma, o uso de sistema computacionais, conjunto de componentes que permitem a entrada, processamento e saída de dados é bastante útil no que tange a gestão de empresas hospitaleiras. Visto que, otimiza e facilita o acesso a dados que constituem informações precisas,  que são importantes para o sucesso do negócio (HOFFMANN; OLIVEIRA; ZEFERINO, 2012).

%------------------------- Bibliografia---------------------%

@book{martins2011,
	author    = {Martins, Cláudia Araújo de Menezes Gonçalves and Gondim, Lorena Regina},
	title     = {Gestão hoteleira},
	publisher = {Centro de Educação Tecnológica do Amazonas},
	address   = {Manaus},
	year      = {2011},
	pages     = {38},
	url       = {https://redeetec.mec.gov.br/images/stories/pdf/eixo_hosp_lazer/061112_gest_hot.pdf},
	note         = {Acesso em: 15 jun. 2025},
}
@techreport{sidonio2015,
	author       = {Sidônio, Letícia Veloso},
	title        = {Gestão hoteleira},
	institution  = {Instituto Federal do Norte de Minas Gerais},
	year         = {2015},
	type         = {Relatório técnico / PDF},
	url          = {https://biblioteca.unisced.edu.mz/bitstream/123456789/2326/1/Gest%C3%A3o%20Hoteleira%20Autor%20Let%C3%ADcia%20Veloso%20Sid%C3%B4nio.pdf},
	note         = {Acesso em: 15 jun. 2025}
}
@article{mauricio2011,
	author  = {Maurício, N. R. and Ramos, K. C. M. de},
	title   = {Gestão na hotelaria},
	journal = {Revista F@pciência},
	year    = {2011},
	volume  = {8},
	number  = {11},
	pages   = {99--113},
	address = {Apucarana – PR},
	issn    = {1984-2333},
	url     = {https://www.fap.com.br/fap-ciencia/edicao_2011/011.pdf},
	note    = {Acesso em: 17 jun. 2025}
}
@article{buhalis1998,
	author    = {Buhalis, Dimitrios},
	title     = {Strategic use of information technologies in the tourism industry},
	journal   = {Tourism Management},
	year      = {1998},
	volume    = {19},
	number    = {5},
	pages     = {409--421},
	issn      = {0261-5177},
	doi       = {10.1016/S0261-5177(98)00038-7},
	url       = {https://www.researchgate.net/publication/222452921_Strategic_Use_of_Information_Technologies_in_the_Tourism_Industry},
}
@article{zeferino2012,
	author  = {Hoffmann, Rosa Cristina and Oliveira, Patrícia Santos Marcondes de and Zeferino, Renato Zanelato},
	title   = {A utilização estratégica dos Sistemas de Informações Gerenciais no ramo hoteleiro da cidade de Ponta Grossa - Paraná},
	journal = {Revista de Engenharia e Tecnologia},
	year    = {2012},
	volume  = {4},
	number  = {1},
	pages   = {18--33},
	issn    = {2176-7270},
	month   = {Abril},
	url     = {https://revistas.uepg.br/index.php/ret/article/view/11290},
	note    = {Acesso em: 17 jun. 2025}
}
@article{zanella2006,
	author  = {Zanella, Liane Carly Hermes and Angeloni, Maria Terezinha},
	title   = {Pousadas - uma alternativa criativa de hospedagem},
	journal = {Turismo - Visão e Ação},
	volume  = {8},
	number  = {2},
	pages   = {253--271},
	year    = {2006},
	month   = {mai–jul},
	publisher = {Universidade do Vale do Itajaí},
	address = {Camboriú, Brasil}
}
